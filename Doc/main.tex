% Template:     Informe LaTeX
% Documento:    Archivo principal
% Versión:      8.2.4 (29/04/2023)
% Codificación: UTF-8
%
% Autor: Pablo Pizarro R.
%        pablo@ppizarror.com
%
% Manual template: [https://latex.ppizarror.com/informe]
% Licencia MIT:    [https://opensource.org/licenses/MIT]

% CREACIÓN DEL DOCUMENTO
\documentclass[
	spanish, % Idioma: spanish, english, etc.
	oneside
]{article}

% INFORMACIÓN DEL DOCUMENTO
\def\documenttitle {Tarea Computacional 2}
\def\documentsubtitle {El problema del Vendedor Viajero y PuLP}
\def\documentsubject {El problema del Vendedor Viajero y su solución mediante Python con PuLP}

\def\documentauthor {Artigues & Meza}
\def\coursename {Optimización I}
\def\coursecode {546351}

\def\universityname {Universidad de Concepción}
\def\universityfaculty {Facultad de Ingeniería}
\def\universitydepartment {Departamento de Informática y Ciencias de la Computación}
\def\universitydepartmentimage {departamentos/fiudec2}
\def\universitydepartmentimagecfg {height=1.57cm}
\def\universitylocation {Concepción, Chile}

% INTEGRANTES, PROFESORES Y FECHAS
\def\authortable {
	\begin{tabular}{ll}
		Integrantes:
		& \begin{tabular}[t]{l}
			Roberto Felipe Artigues Escobar \\
			Emilio Juan Meza Quiroz
		\end{tabular} \\ & \\
		Profesora:
		& \begin{tabular}[t]{l}
			Rosa Medina
		\end{tabular} \\
		% Auxiliar:
		% & \begin{tabular}[t]{l}
		% 	Auxiliar 1
		% \end{tabular} \\
		% Ayudantes:
		% & \begin{tabular}[t]{l}
		% 	Ayudante 1 \\
		% 	Ayudante 2
		% \end{tabular} \\
		% \multicolumn{2}{l}{Ayudante de laboratorio: Ayudante 1} \\
		& \\
		% \multicolumn{2}{l}{Fecha de realización: \today} \\
		\multicolumn{2}{l}{Fecha de entrega: 10 de Noviembre de 2023} \\
		\multicolumn{2}{l}{\universitylocation}
	\end{tabular}
}

\DeclareUnicodeCharacter{0301}{*************************************}

% IMPORTACIÓN DEL TEMPLATE
\input{template}

% INICIO DE PÁGINAS
\begin{document}

% PORTADA
\templatePortrait

% CONFIGURACION DE PÁGINA Y ENCABEZADOS
\templatePagecfg

% RESUMEN O ABSTRACT
% \begin{abstractd}
% 	% \lipsum[1] % Párrafo ejemplo, se puede borrar
% 	Se realizó un trabajo experimental en el cual se aplicaron cargas de distintos pesos a una armadura de tipo Warren. Durante el proceso, se midieron los valores de deformación utilizando un medidor de deformaciones y se mantuvieron registrados mientras se mantenían las cargas aplicadas. \\ \\
% 	A continuación, se multiplicaron los valores obtenidos por la constante 𝛼 con el fin de obtener los valores de fuerza correspondientes. Además, se pudo determinar si las barras de la armadura estaban sometidas a tracción o compresión.
% 	\\ \\
% 	Los objetivos del trabajo fueron:  
% 	\begin{itemize}
% 		\item Estudiar el equilibrio de fuerzas en una estructura.
% 		\item Medir fuerza axial en miembros estructurales.
% 		\item Hacer una breve descripción del banco de ensayos, condiciones de carga.
% 		\item Indicar el diagrama de cuerpo libre de la armadura.
% 		\item Realizar una comparación y discusión de los resultados.
% 	\end{itemize}

% 	\noindent Con respecto a los resultados obtenidos no fueron los esperados, tanto en magnitud de fuerzas como en estados de carga. 
% 	Esto pudo haber ocurrido por fallos en los materiales o por errores al momento de medir las cargas. \\

% 	\noindent En conclusión, es posible afirmar que el análisis teórico solo representa una aproximación a la realidad y no tiene en cuenta todos 
% 	los factores que podrían influir en los resultados de la experimentación en condiciones reales, por lo tanto, cabe la posibilidad 
% 	de que los resultados del análisis teórico y experimental no sean iguales.
	 

% \end{abstractd}

% TABLA DE CONTENIDOS - ÍNDICE
% \templateIndex

% CONFIGURACIONES FINALES
\templateFinalcfg

% ======================= INICIO DEL DOCUMENTO =======================

\section{Contexto del Problema}

En el año 2150, la humanidad ha establecido colonias en diversos sistemas estelares. La empresa "Courier Cósmico S.A." se enfrenta al desafío de optimizar las rutas de sus naves espaciales para el abastecimiento de estas colonias. El costo de viajar entre colonias es asimétrico debido a factores como las diferencias gravitacionales, corrientes espaciales y eventos cósmicos. Este problema se conoce como el "Problema del Viajante Espacial" (PVE).
\vspace{12pt}

\noindent El objetivo es diseñar una ruta que permita a una nave visitar cada colonia una sola vez y regresar a la base terrestre de la manera más eficiente posible. La nave debe llevar suministros vitales, realizar mantenimientos y transportar pasajeros, todo esto bajo la restricción de minimizar los costos totales que incluyen consumo de combustible, tiempo, peajes espaciales y maniobras de asistencia gravitatoria.
\vspace{12pt}



\newpage

\section*{Modelación del problema}

En el contexto de la Optimización de Rutas Espaciales Intercoloniales (OREI), consideramos el conjunto de colonias interstelares que deben ser visitadas exactamente una vez por una nave espacial, que parte y regresa a la base terrestre. El costo \( c_{ij} \) representa el consumo de combustible, tiempo, peajes espaciales y maniobras gravitatorias entre la colonia \( i \) y la colonia \( j \), y no es simétrico debido a las diferentes condiciones espaciales.

\subsection*{Formulación de Dantzig-Fulkerson-Johnson (DFJ)}

Función Objetivo:
\begin{equation}
    \min \sum_{i=1}^{n}\sum_{j=1}^{n} c_{ij} x_{ij}
\end{equation}

Sujeto a:
\begin{align}
    \sum_{i=1}^{n} x_{ij} &= 1, && \text{para toda colonia } j \\
    \sum_{j=1}^{n} x_{ij} &= 1, && \text{para toda colonia } i \\
    \sum_{i \in S}\sum_{j \in S} x_{ij} &\leq |S| - 1, && \text{para cualquier subconjunto } S \text{ de colonias, con } S \neq \emptyset \\
    x_{ij} &\in \{0, 1\}, && \text{si la nave viaja de } i \text{ a } j
\end{align}

\subsection*{Formulación de Miller-Tucker-Zemlin (MTZ)}

Incluye todas las restricciones de la formulación DFJ y añade:

\begin{equation}
    u_i - u_j + (n - 1)x_{ij} \leq n - 2, \quad \text{para } i, j = 2, \ldots, n
\end{equation}

\noindent Donde \( u_i \) es una variable continua que representa el orden de visita a la colonia \( i \) en la ruta, ayudando a prevenir subrutas dentro del recorrido.

\subsection*{Formulación de Gillett-Gomory (GG)}

Incluye todas las restricciones de la formulación DFJ y añade:

\begin{align}
    \sum_{j=1}^{n} g_{ij} - \sum_{j=2}^{n} g_{ji} &= 1, \text{\indent para toda colonia } i \text{ excepto la base terrestre} \\
    0 \leq g_{ij} \leq (n - 1)x_{ij}, && \text{si hay flujo de la nave entre } i \text{ y } j
\end{align}

donde \( g_{ij} \) son variables de flujo que representan el número de naves entre las colonias \( i \) y \( j \), asegurando la conservación del flujo y la no formación de subrutas.


% \section{Situación Propuesta}



% En una galaxia muy, muy lejana, el Gran Moff Tarkin se enfrenta al desafío de mantener el control del Imperio Galáctico. Su misión es optimizar los recursos imperiales para maximizar el control sobre la galaxia mientras se minimiza cualquier forma de insurrección rebelde. Con un enfoque multifacético que abarca desde la fuerza militar hasta la diplomacia, Tarkin debe tomar decisiones que afectarán el destino del Imperio.


% \section{Variables de Decisión}



% Las siguientes variables de decisión se consideran en el modelo:



% \begin{itemize}

%     \item \( x_1 \) = Número de Destructores Estelares construidos

%     \item \( x_2 \) = Número de Stormtroopers entrenados

%     \item \( x_3 \) = Número de cazas TIE fabricados

%     \item \( x_4 \) = Número de planetas bajo vigilancia directa

%     \item \( x_5 \) = Cantidad de recursos asignados para espionaje

%     \item \( x_6 \) = Cantidad de recursos asignados para propaganda

%     \item \( x_7 \) = Número de alianzas con cárteles de comercio

%     \item \( x_8 \) = Número de Jedi capturados

%     \item \( x_9 \) = Número de misiones diplomáticas realizadas

%     \item \( x_{10} \) = Número de Ewoks reclutados como mascotas para la moral

% \end{itemize}

% \newpage

% \section{Función Objetivo y justificación}
%  \newblock

% Maximizar \( Z = 3x_1 + 2x_2 + x_3 + 4x_4 - x_5 + 2x_6 + x_7 + 5x_8 + x_9 - 3x_{10} \) \\


% \noindent La función objetivo busca maximizar el control imperial sobre la galaxia, y los coeficientes asignados a cada variable reflejan su importancia relativa en esta tarea. A continuación, se detallan las razones para la elección de cada coeficiente:

% \begin{itemize}
%     \item \textbf{Destructores Estelares \( (x_1) \)}: Coeficiente 3 \\
%     Vital para el control de sistemas estelares; representa el poderío militar del Imperio.
    
%     \item \textbf{Stormtroopers \( (x_2) \)}: Coeficiente 2 \\
%     Importantes para la fuerza terrestre pero no tan críticos como los Destructores Estelares.
    
%     \item \textbf{Cazas TIE \( (x_3) \)}: Coeficiente 1 \\
%     Útiles pero fácilmente producibles y menos críticos.
    
%     \item \textbf{Planetas bajo vigilancia \( (x_4) \)}: Coeficiente 4 \\
%     Cruciales para prevenir rebeliones y mantener el orden.
    
%     \item \textbf{Espionaje \( (x_5) \)}: Coeficiente -1 \\
%     Útil pero potencialmente dañino si se descubre.
    
%     \item \textbf{Propaganda \( (x_6) \)}: Coeficiente 2 \\
%     Ayuda a mantener la moral alta y el apoyo público.
    
%     \item \textbf{Alianzas con cárteles \( (x_7) \)}: Coeficiente 1 \\
%     Útiles pero menos confiables que los recursos internos.
    
%     \item \textbf{Jedi capturados \( (x_8) \)}: Coeficiente 5 \\
%     Un golpe significativo para la Rebelión y un gran logro para el Imperio.
    
%     \item \textbf{Misiones diplomáticas \( (x_9) \)}: Coeficiente 1 \\
%     Útiles pero no garantizan lealtad a largo plazo.
    
%     \item \textbf{Ewoks \( (x_{10}) \)}: Coeficiente -3 \\
%     Potencial fuente de rebelión; se manejan con cuidado.
% \end{itemize}

% \newpage

% \section{Restricciones y Justificación}

% Las restricciones del modelo se enumeran a continuación:

% \[
% \begin{aligned}
%     x_1 + x_3 & \leq 100 \quad & (1) \\
%     x_2 & \geq 500 \quad & (2) \\
%     x_5 + x_6 & \leq 50 \quad & (3) \\
%     x_4 + x_9 & \leq 10 \quad & (4) \\
%     x_{10} & \leq \frac{x_2}{10} \quad & (5) \\
%     x_8 & \leq 50 \quad & (6) \\
%     x_7 & \leq 50 \quad & (7) \\
%     x_2 & \leq (x_1 + x_3) \times 10 \quad & (8)
% \end{aligned}
% \]

% A continuación, se detallan las justificaciones para cada una de las restricciones:

% \begin{enumerate}
%     \item Capacidad de producción militar: Esta restricción se debe a limitaciones en las fábricas de Kuat Drive Yards y Sienar Fleet Systems, que solo pueden producir 100 unidades de naves grandes y cazas TIE en total.
%     \item Mínimo de Stormtroopers para mantener el orden: De acuerdo con análisis históricos de conflictos internos, se ha determinado que al menos 500 Stormtroopers son necesarios para mantener la paz y el orden en los territorios del Imperio.
%     \item Presupuesto para espionaje y propaganda: El Senado Imperial ha asignado un máximo de 50 millones de créditos galácticos para estas actividades.
%     \item Límite de operaciones planetarias: Debido a la escasez de líderes competentes y diplomáticos capacitados, el Imperio solo puede manejar 10 operaciones planetarias a la vez.
%     \item No más Ewoks que una décima parte de Stormtroopers: Los Ewoks son difíciles de manejar y requieren una supervisión constante. Por lo tanto, el número de Ewoks no debe superar una décima parte del número de Stormtroopers para evitar problemas de manejo.
%     \item Número Máximo de Jedi Capturados: Solo hay 50 bases en el Imperio capaces de albergar y contener a un Jedi.
%     \item Número Máximo de Alianzas con Cárteles: Hay un número limitado de bases planetarias que pueden gestionar alianzas con cárteles de comercio.
%     \item Relación entre Stormtroopers y Naves: Cada nave (Destructor Estelar o caza TIE) requiere un cierto número de Stormtroopers para operar eficazmente.
% \end{enumerate}

% \section{Resultados y Discusión}

% \subsection{Resultados Excel}
% Los resultados obtenidos a través del solver de Excel ofrecen un valor objetivo de \( Z = 2800 \).

% \begin{table}[h]
%     \centering
%     \begin{tabular}{|c|c|c|}
%         \hline
%         Variable & Coeficiente & Valor \\
%         \hline
%         \( x_1 \) & 3 & 100 \\
%         \( x_2 \) & 2 & 1000 \\
%         \( x_3 \) & 1 & 0 \\
%         \( x_4 \) & 4 & 50 \\
%         \( x_5 \) & -1 & 0 \\
%         \( x_6 \) & 2 & 50 \\
%         \( x_7 \) & 1 & 50 \\
%         \( x_8 \) & 5 & 10 \\
%         \( x_9 \) & 1 & 100 \\
%         \( x_{10} \) & -3 & 0 \\
%         \hline
%     \end{tabular}
%     \caption{Resultados obtenidos con Excel}
% \end{table}

% \subsection{Resultados AMPL}
% El modelo se resolvió con éxito utilizando el solver CPLEX, alcanzando una solución óptima con un valor objetivo de \( Z = 2740 \).

% \begin{table}[h]
%     \centering
%     \begin{tabular}{|c|c|c|}
%         \hline
%         Variable & Coeficiente & Valor \\
%         \hline
%         \( x_1 \) & 3 & 100 \\
%         \( x_2 \) & 2 & 1000 \\
%         \( x_3 \) & 1 & 0 \\
%         \( x_4 \) & 4 & 10 \\
%         \( x_5 \) & -1 & 0 \\
%         \( x_6 \) & 2 & 50 \\
%         \( x_7 \) & 1 & 50 \\
%         \( x_8 \) & 5 & 50 \\
%         \( x_9 \) & 1 & 0 \\
%         \( x_{10} \) & -3 & 0 \\
%         \hline
%     \end{tabular}
%     \caption{Resultados obtenidos con AMPL}
% \end{table}

% \noindent A pesar de las diferencias en los resultados, ambos modelos sugieren un fuerte enfoque en la construcción de Destructores Estelares y el entrenamiento de Stormtroopers. Sin embargo, el modelo de Excel asigna más recursos a operaciones de vigilancia planetaria y misiones diplomáticas, lo que podría indicar diferencias en la implementación o en los parámetros del solver.























% \input{example} % Ejemplo, se puede borrar
% \section{Introducción}
% El logro del equilibrio en las estructuras es fundamental para garantizar su estabilidad y resistencia. En este informe, nos centraremos en el estudio y análisis de las armaduras, que son estructuras compuestas por una serie de miembros conectados entre sí, formando una configuración rígida. Es importante destacar que, cuando los miembros de la armadura se encuentran en un mismo plano, se les denomina armaduras planas. En este caso, nos enfocaremos específicamente en una estructura de tipo Warren, reconocida por su característica forma triangular. \\ \\ 
% En el presente experimento, llevaremos a cabo pruebas utilizando diferentes cargas que serán cuidadosamente medidas utilizando un dinamómetro. Para obtener datos precisos sobre la deformación producida por cada carga, utilizaremos galgas extensiométricas (strain gauges). Estas galgas nos permitirán medir el valor de la deformación ($\epsilon$) en cada carga, lo cual nos proporcionará la información necesaria para calcular la fuerza ($\vec{F}$). Es importante tener en cuenta que la relación entre la fuerza y la deformación se establece mediante una constante de calibración $\alpha$, por lo que se realizará un proceso de calibración antes de obtener los resultados finales. \\ \\
% Para el análisis de los resultados y la resolución de los cálculos, emplearemos el método de Nodos, que establece que si una estructura se encuentra en equilibrio, entonces cada una de sus uniones también lo estará. A través de este enfoque, podremos aplicar nuestros conocimientos sobre el equilibrio de estructuras y estudiar el equilibrio de fuerzas en la armadura Warren en particular. Uno de nuestros objetivos principales será medir la fuerza axial en los diferentes miembros estructurales que componen la armadura. \\ \\
% Con este informe, buscamos no solo comprender los principios teóricos detrás del equilibrio de estructuras y la aplicación de las armaduras, sino también adquirir experiencia práctica en la medición de fuerzas y la interpretación de los resultados obtenidos. A través de este proceso, esperamos obtener una visión más profunda de los conceptos fundamentales de la mecánica estructural y su aplicación en el diseño y análisis de estructuras resistentes y eficientes. 

% \newpage
% \section{Desarrollo}
% \subsection{Análisis experimental}
% 	Para el desarrollo del laboratorio se utilizaron los siguientes materiales:
% 	\begin{multicols}{2}
% 		\begin{itemize}
% 			\item Armadura tipo Warren.
% 			\item 4 galgas extensiométricas.
% 		\end{itemize}
% 		\begin{itemize}
% 			\item Dinamómetro análogo de 25 Kg.  
% 			\item Medidor de deformaciones.
% 		\end{itemize}
% 	\end{multicols}

% 	\noindent Se llevaron a cabo cuatro ensayos, en los cuales se aplicó una fuerza horizontal en el nodo H,
% 	midiendo las deformaciones tanto durante la aplicación de la fuerza como posteriormente, 
% 	estando en reposo. Las fuerzas aplicadas, de 5, 10, 15 y 20 kg, se midieron utilizando el dinamómetro
% 	analogo.

% 	\begin{table}
% 		\centering		
% 		\begin{tabular}{lllll}
% 			\cline{2-5}
% 						& \multicolumn{2}{l}{Barra 1} & \multicolumn{2}{l}{Barra 4} \\ \hline
% 			Fuerza (Kg) & Carga        & Reposo       & Carga        & Reposo       \\ \hline
% 			5           & 12           & -2           & 7            & 13           \\
% 			10          & 16           & -5           & 10           & 14           \\
% 			15          & 28           & -6           & 10           & 14           \\
% 			20          & 29           & 5            & 3            & 11           \\ \hline
% 			\end{tabular}
% 		\label{tab:tabla-mediciones}
% 		\caption{Medición de micro strain en las barras 1 y 4.}
% 	\end{table}
	
	
	
% 	\begin{center}
% 		Para calcular los siguientes valores, se utilizaron los siguientes datos:
% 	\end{center}
% 	\begin{align*}
% 		\epsilon &= \text{Carga} - \text{Reposo}, & \vec{F} &= \alpha \epsilon
% 	\end{align*}

% 	\begin{table}
% 		\centering
% 		\begin{tabular}{lllll}
% 		\cline{2-5}
% 					& \multicolumn{2}{l}{Barra 1} & \multicolumn{2}{l}{Barra 2} \\ \hline
% 		Fuerza (Kg) & $\epsilon$ & $\vec{F}$ & $\epsilon$ & $\vec{F}$   \\ \hline
% 		5           & 14         & 4.3652    & -6         & -1.8708   \\
% 		10          & 21         & 6.5478    & -4         & -1.2472   \\
% 		15          & 34         & 10.6012   & -4         & -1.2472   \\
% 		20          & 24         & 7.4832    & -8         & -2.4944   \\ \hline
% 		\end{tabular}
% 		\caption{Fuerzas experimentales en las barras 1 y 4.}
% 		\label{tab:tabla-fuerzas}
% 	\end{table}

% \newpage
% \subsection{Análisis teórico}
% 	Para poder obtener los valores teóricos en el problema, hay que definir ciertas características
% 	para poder trabajar en la armadura Warren El análisis de fuerza se realizará en un plano 
% 	bidimensional y dada la distribución de fuerzas, para realizar los cálculos se considerará lo siguiente dado la simetría de la armadura:

% 	\begin{equation*}
% 		\vec{F}' = \frac{\vec{F}}{2}
% 	\end{equation*}

% 	\begin{figure}[ht]
% 		\centering
% 		\includegraphics[width=0.8\textwidth]{img/dcl_armadura.png}
% 		\caption{DCL de la armadura Warren. }
% 		% \subcaption{$F =$ Fuerza \\ $A = $ Pasador \\ $G = $ Rodillo }
% 		\label{fig:dcl_armadura}
% 	\end{figure}

% 	\noindent \\  Para obtener los cálculos teóricos de las fuerzas desconocidas se realiza un 
% 	equilibrio de fuerzas en X e Y, y un momento aplicado al nodo A.
% 	% math block
% 	\begin{align*}
% 		\sum F_x &= 0 \\
% 		\sum F_y &= 0 \\
% 		\sum M_A &= 0
% 	\end{align*} 

% 	\noindent Resolviendo estas ecuaciones, se obtiene:
% 	\begin{align*}
% 		\Sigma F_x &= 0  &&\Rightarrow  A_x - F = 0  &&\Rightarrow  A_x  = F \\
% 		\Sigma F_y &= 0  &&\Rightarrow  A_y + G_y = 0  &&\Rightarrow  A_y = -G_y \\
% 		\Sigma M_A &= 0 &&\Rightarrow 18 F + 108 G_y = 0  &&\Rightarrow G_y = -\frac{18F}{108}
% 	\end{align*}

% 	A continuación, se determinan las fuerzas internas en las barras utilizando el método de nodos.

% 	% Analisis del nodo A
% 	\pagebreak
% 	\textbf{Nodo A:}
% 	\begin{figure}[ht]
% 		\centering
% 		\includegraphics[width=0.18\textwidth]{img/dcl_A.png}
% 		\caption{DCL del nodo A.}
% 		% \subcaption{$F =$ Fuerza \\ $A = $ Pasador \\ $G = $ Rodillo }
% 		\label{fig:dcl_nodoA}
% 	\end{figure}

% 	Se aplica el equilibrio de fuerzas en las direcciones X e Y, involucrando las fuerzas $A_x$, $F_{AB}$, $F_{AH}$ y $A_y$:

% 	\begin{align*}
% 		\Sigma F_x &= 0: & A_x - F_{AB} - F_{AH}\cos(45) &= 0 \\
% 		\Sigma F_y &= 0: & A_y - F_{AH}\sin(45) &= 0
% 	\end{align*}

% 	\noindent Resolviendo estas ecuaciones, se halla la fuerza en la barra AH ($F_{AH}$):

% 	\begin{equation*}
% 		F_{AH} = \frac{\sqrt{2}}{6} \cdot F
% 	\end{equation*}

% 	\noindent El resultado positivo indica que $F_{AH}$ está en compresión. \\

% 	% Analisis del nodo B
% 	\textbf{Nodo B:}
% 	\begin{figure}[ht]
% 		\centering
% 		\includegraphics[width=0.2\textwidth]{img/dcl_B.png}
% 		\caption{DCL del nodo B.}
% 		% \subcaption{$F =$ Fuerza \\ $A = $ Pasador \\ $G = $ Rodillo }
% 		\label{fig:dcl_nodoB}
% 	\end{figure}

% 	Se aplica el equilibrio de fuerzas en las direcciones X e Y, involucrando las fuerzas $F_{AB}$, $F_{BH}$ y $F_{BC}$:

% 	\begin{align*}
% 		\Sigma F_x &= 0: & F_{AB} - F_{BH}\cos(45) - F_{BC} &= 0 \\
% 		\Sigma F_y &= 0: & F_{BH}\sin(45) &= 0
% 	\end{align*}

% 	A partir de la ecuación $\Sigma F_y = 0$, se deduce que $F_{BH} = 0$. Posteriormente, utilizando la ecuación $\Sigma F_x = 0$, se determina que $F_{AB}$ y $F_{BC}$ son iguales y opuestas:

% 	\begin{equation*}
% 		F_{AB} - F_{BC} = 0 \Rightarrow F_{AB} = F_{BC}
% 	\end{equation*}

% 	Para calcular las fuerzas en las barras $BC$ y $AH$, se sustituye la relación $F_{AB} = F_{BC}$ en la ecuación del equilibrio de fuerzas en X en el nodo A:

% 	\begin{equation*}
% 		F_{AB} = F - \frac{\sqrt{2}}{6} \cdot F \cdot \cos(45)
% 	\end{equation*}

% 	Simplificando la ecuación, se obtienen las fuerzas en las barras 1 ($F_{AB}$) y 2 ($F_{BC}$):

% 	\begin{equation*}
% 		F_{AB} = F_{BC} = F \cdot \frac{5}{6}
% 	\end{equation*}

% 	\noindent Así, mediante el análisis teórico y la aplicación del método de nodos en los nodos A y B de la armadura Warren bidimensional, se han determinado las fuerzas internas en las barras AH y BC. \\ \\
% 	La barra AH presenta una fuerza interna de $F_{AH} = \frac{\sqrt{2}}{6} \cdot F$, indicando que está en compresión. Por otro lado, las barras AB y BC tienen fuerzas internas iguales, $F_{AB} = F_{BC} = F \cdot \frac{5}{6}$, también en compresión.
% 	\begin{table}[ht]
% 		\centering
% 		\caption{Tabla de Fuerzas teóricas de la Barra 1 y 4}
% 		\begin{tabular}{|c|c|c|}
% 			\hline
% 			Fuerza F (kg) & Barra (1) & Barra (4) \\
% 			\hline
% 			5 & 20,83 N & 5,89 N \\
% 			10 & 41,67 N & 11,79 N \\
% 			15 & 62,50 N & 17,68 N \\
% 			20 & 83,3 N & 23,57 N \\
% 			\hline
% 		\end{tabular}
% 	\end{table}

% \newpage
% \subsection{Comparación y discusión de resultados}

% 	\begin{table}[ht]
% 		\centering
% 		\caption{Comparación de resultados de la Barra 1}
% 		\begin{tabular}{cccccc}
% 			\textbf{Fuerza} & \textbf{Estado Carga} & \textbf{F.Experimental} & \textbf{F.Teorica} & \textbf{\%Error} \\
% 			\hline
% 			5 & Compresión & 4.3652 N & 20.83 N & 79\% \\
% 			10 & Compresión & 6.5478 N & 41.67 N & 84\% \\
% 			15 & Compresión & 10.6012 N & 62.50 N & 83\% \\
% 			20 & Compresión & 7.4832 N & 83.3 N & 91\% \\
% 		\end{tabular}
% 	\end{table}

% 	\begin{table}[ht]
% 		\centering
% 		\caption{Comparación de resultados de la Barra 4}
% 		\begin{tabular}{cccccc}
% 			\textbf{Fuerza} & \textbf{Estado Carga} & \textbf{F.Experimental} & \textbf{F.Teorica} & \textbf{\%Error} \\
% 			\hline
% 			5 & Compresión & -1.8708 N & 5.89 N & 68\% \\
% 			10 & Compresión & -1.2472 N & 11.79 N & 89\% \\
% 			15 & Compresión & -1.2472 N & 17.68 N & 93\% \\
% 			20 & Compresión & -2.4944 N & 23.57 N & 89\% \\
% 		\end{tabular}
% 	\end{table}

% 	\textbf{Para la Barra 1:}
% 	\begin{itemize}
% 	\item Existe una diferencia significativa entre los valores experimentales y teóricos de la fuerza en todas las cargas evaluadas.
% 	\item El porcentaje de error oscila entre el 79\% y el 91\%.
% 	\item Los valores experimentales de fuerza son inferiores a los valores teóricos en todas las cargas.
% 	\item La discrepancia entre los valores experimentales y teóricos indica posibles errores o imprecisiones en las mediciones o en los cálculos teóricos utilizados.
% 	\end{itemize}

% 	\newpage
% 	\textbf{Para la Barra 4:}
% 	\begin{itemize}
% 	\item Al igual que en la Barra 1, se observa una diferencia considerable entre los valores experimentales y teóricos de la fuerza en todas las cargas evaluadas.
% 	\item El porcentaje de error varía entre el 68\% y el 93\%.
% 	\item Los valores experimentales de fuerza son negativos en todas las cargas, lo que indica una dirección opuesta a la esperada.
% 	\item La discrepancia entre los valores experimentales y teóricos sugiere la presencia de errores sistemáticos o aleatorios en las mediciones o en el modelo teórico utilizado.
% 	\end{itemize}



% 	\noindent En general, estas conclusiones revelan que los datos experimentales obtenidos para ambas barras presentan desviaciones significativas en comparación con los valores teóricos esperados. Estos resultados indican la necesidad de revisar y mejorar los métodos de medición y análisis utilizados en el estudio, con el objetivo de reducir los errores y obtener resultados más precisos y consistentes en futuras investigaciones.

% \newpage
% \section{Conclusión}
% Se sometieron las barras de una armadura tipo Warren a diversas fuerzas, tanto de tracción como de compresión, y se observó su comportamiento. Posteriormente, se realizó un análisis experimental que arrojó ciertos resultados, los cuales se compararon en una tabla con los obtenidos en el análisis teórico. 
% \\ \\
% \noindent Sin embargo, al contrastar los datos obtenidos en ambos análisis, se notó que no coincidían. Esto pudo deberse a varias razones, como por ejemplo, la mala calibración de los sensores de peso durante el experimento en el laboratorio, o que la armadura no estuviera en las condiciones adecuadas para llevar a cabo la prueba.
% \\ \\
% \noindent Dejando lo anterior de lado, se logró estudiar adecuadamente el equilibrio de las fuerzas en una estructura, además de realizar un análisis teórico acerca de la misma estructura utilizando métodos vistos en clases.
% \\ \\
% \noindent En resumen, el estudio realizado contribuye al entendimiento de los principios básicos del balance de fuerzas en estructuras, y destaca la importancia de la precisión y rigor en el análisis experimental y teórico en el campo de la ingeniería. Aunque los resultados obtenidos no coincidieron exactamente, este trabajo sienta las bases para futuras investigaciones y análisis.


% \newpage





% \section{Referencias}
% \begin{itemize}
%     \item Ferdinand P. Beer, E. Russel Johnston y Elliot R. Eisenberg. (1998). Mecánica vectorial para ingenieros. Estática y Dinámica. Mc Graw-Hill. ISBN 0070053677 y 9701019512.
%     \item Convert Units – Measurement Unit Converter. (s.f.). Convert kilogram-force to newton. \url{https://www.convertunits.com/from/kilogram-force/to/newton#:~:text=1%20kilogram%2Dforce%20is%20equal%20to%209.80665%20newton}
% \end{itemize}

% FIN DEL DOCUMENTO
\end{document}